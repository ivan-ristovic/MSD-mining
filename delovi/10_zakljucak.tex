\section{Zaključak}
\label{sec:Zakljucak}

Dobijeni rezultati se oslanjaju na podskup skupa podataka koji u sebi sadr\v{z}i milion pesama. Kori\v{s}\'c{}enje celog skupa podataka bi donelo jo\v{s} pouzdanije rezultate. Medjutim, i sam podskup podataka je bio dovoljan da se uo\v{c}e prethodno navedeni zaklju\v{c}ci.

Rezultati imaju prakti\v{c}nu primenu. Uo\v{c}avanje zavisnosti muzi\v{c}kih \v{z}anrova i godina njihove popularnosti, ili njihove popularnosti na razli\v{c}itim podnevljima bi se moglo iskoristiti za automatsko generisanje lista pesama. Ovo bi predstavljao zanimljiv i koristan dodatak za muzi\v{c}ki plejer.

Sajtovi kao \v{s}to su YouTube \cite{youtube} koriste prethodno prikupljene podatke o pesmama. Njih, u kombinaciji sa saznanjem o korisnikovim prethodno preslu\v{s}anim pesmama, koristi kako bi korisniku pru\v{z}io \v{s}to bolje preporuke za slede\'c{}u pesmu, a samim tim ga i zadr\v{z}ao na sajtu. Implementacija sli\v{c}nog, ali dosta jednostavnijeg, sistema bila bi mogu\'c{}a dobijenim rezultatima koji su predstavljeni u ovom radu.
