\section{Preprocesiranje i vizualizacija podataka}
\label{sec:Preprocesiranje}

U ovom odeljku \'c{}emo poku\v{s}ati da \v{c}itaoca upoznamo sa skupom podataka. Uz vizualne prikaze raznovrsnosti skupa i analize njegovih specifi\v{c}nosti do\v{s}li smo do bitnih zaklju\v{c}aka koji su kasnije uticali na dalje istra\v{z}ivanje skupa i njegovih karakteristika.

Pre samog preprocesiranja podataka koje je neophodno za istra\v{z}ivanje, izvr\v{s}ili smo analizu statisti\v{c}kih podataka dobijenih na osnovu skupa. Dobijene statistike se mogu videti u dodatku \ref{sec:Statistika}.

Neki od atributa koji su vizuelizovani kasnije u ovom odeljku nisu u potpunosti prisutni u skupu, tako da je analiza takvih atributa radjena samo nad slogovima gde nema nedostaju\'c{}ih vrednosti za te atribute.

Jedna od op\v{s}tih transformacija je nad atributom koji sadr\v{z}i informacije o \v{z}anru. \v{Z}anr je podatak koji je originalno dat kao niz niski. Medjutim, sadr\v{z}aj ovog niza nije ta\v{c}no definisan, ve\'c{} on ponekad u sebi sadr\v{z}i \v{c}itavu re\v{c}enicu koja opisuje \v{z}anr. Jednostavna, a neophodna transformacija je bila da iz ovog niza izbacimo pojavljivanje re\v{c}i \emph{and}, \v{c}ije je \v{c}esto pojavljivanje remetilo rezultate.
